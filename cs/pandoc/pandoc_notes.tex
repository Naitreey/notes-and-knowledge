% arara: xelatex: {action: nonstopmode,synctex: yes} 
% arara: biber 
% arara: xelatex: {action: nonstopmode,synctex: yes} 
% arara: xelatex: {action: nonstopmode,synctex: yes}
\RequirePackage[l2tabu, orthodox]{nag}
\documentclass{article}

%文献宏包 
%\usepackage[style=numeric,backend=biber]{biblatex}

\usepackage[a4paper]{geometry}

%中文宏包
%\usepackage[UTF8,winfonts,fancyhdr,hyperref,fntef]{ctex}
\usepackage{xeCJK}

\usepackage[pdfborder={0 0 0},colorlinks=false]{hyperref}
\usepackage{graphicx,amsmath,amssymb,bm,mathtools,subcaption,caption,cleveref}

%数学定义

%single & double quotes
\newcommand{\sq}[1]{`#1'}
\newcommand{\dq}[1]{``#1''}

%文献源 \addbibresource{•}

%设置字体
%\setmainfont[Ligatures=TeX]{Helvetica}
%\setsansfont{Helvetica}%Arial
%\setmonofont{}
\setCJKmainfont[BoldFont={SimHei},ItalicFont={楷体}]{SimSun}%Hiragino Sans GB W3/W6
%\setCJKsansfont{Hiragino Sans GB W3}
\setCJKmonofont{幼圆}

%代码宏包
%\usepackage{listings}
%\lstset{language=,frame=single,texcl=true,mathescape=true}%[LaTeX]TeX or mathematica or etc?

\title{Pandoc Notes}
\author{Naitree Zhu}
\date{Last modified \today}

\begin{document}
\maketitle
%目录 \tableofcontents

Verify \textsf{pandoc} is installed and to check its version: type
\verb|pandoc --version| in terminal.

在 terminal 中输入 \verb|pandoc| 打开程序, 再输入所有需要转换的代码或文字, 然后在 windows 下 \verb|<Ctrl-z>| (在 Linux or Mac OS 下 \verb|<Ctrl-d>|) 就得到了转换后的输出.

\verb|pandoc| 默认从 markdown 转换为 HTML 4. 设置默认转换格式: \verb|pandoc -f xx -t xxx| 则从 xx 转换为 xxx.

将一种格式的文件转换为另一种格式的文件的基本操作:

\begin{verbatim}
pandoc file1.a -f xx -t xxx -s -o file2.b
\end{verbatim}

The \verb|-s| flag tells \verb|pandoc| to create a \dq{standalone} file and \verb|-o| flag tells to put the output in the file \verb|file2.b|. Note that \verb|-f xx| and \verb|-t xxx| can be \emph{omitted} if \verb|a| and \verb|b| are filename extensions of corresponding formats. So \verb|pandoc file1.a -s -o file2.b| could be sufficient.

For creating a PDF directly, a \LaTeX~distribution must be installed in the operating system.

\verb|pandoc --help| to get a list of all the supported options.

type \verb|man pandoc| on Linux or Mac OS to get \textsf{pandoc} manual page and \verb|man pandoc_markdown| to get a description of pandoc's markdown syntax.




\end{document}


