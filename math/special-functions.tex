\documentclass{article}
\usepackage{physics}
% \dv[n]{y}{x} 导数
% \pdv[n]{z}{x}{y} 偏导
% \fdv[n]... 变分

\begin{document}
%%%%%%%%%%%%%%%%%%%%
\section{Heaviside step function}
%%%%%%%%%%%%%%%%%%%%
\begin{itemize}
    \item 斜坡函数的导函数.
        \begin{equation}
            H(x) = \dv{max{x,0}}{x}
        \end{equation}
    \item delta function 的原函数.
        \begin{equation}
            H(x) = \int_{-\infty}{x}\delta(s)ds
        \end{equation}
\end{itemize}

\begin{equation}
    R(x) = max(x, 0)
\end{equation}

%%%%%%%%%%%%%%%%%%%%
\section{Ramp function}
%%%%%%%%%%%%%%%%%%%%
\begin{itemize}
    \item 形状像 ramp, 即斜坡.
    \item
        由于负值输入时输出为 0, 非负输入时输出等价于输入的性质, 在
        电路中作为整流函数, 将交流电转变为直流电.
    \item
        在神经网络中作为整流函数, 类似电路中作用.
    \item
        其导数是 Heaviside step function.
        \begin{equation}
            R'(x) = H(x) \text{for} x \ne 0
        \end{equation}
        二阶导数是 Dirac delta function, 从而 ramp function 是 Green's function.
        \begin{equation}
            \dv[2]{R(x-x_0)}{x} = \delta(x-x_0)
        \end{equation}
\end{itemize}
\end{document}
