\documentclass{article}
\begin{document}
\section{电磁感应}
从各种场的起源可以很容易理解不同场的有源性和无源性:
\begin{enumerate}
  \item 库伦电场 (静电场) 是由电荷产生的, 它有直接的物质基础, 因此有源;
  \item 静磁场是由电流产生的, 或者说是由电荷运动这个行为所产生的, 无直接的物质基础, 故无源;
  \item 感生电场和感生磁场都是由于另一种场的变化而产生的, 也没有直接的物质基础, 故无源.
\end{enumerate}
可以看出有源场是有直接物质基础的, 因此相对更本质; 无源场则是因为某种行为或现象衍生出来的. 但磁场是与电场同样重要的场, 研究磁场的意义并没有因它的衍生性减弱, 因为磁场的真实性和相对于电场的不同特性并没有因此而不同 (我们确实在同一座标系下分别观测到了这两种性质迥异的场).
\section{Q \& A}
\begin{question}
  带电粒子进入匀强磁场中受 Lorentz 力为什么一定沿圆周轨迹运动?
\end{question}
\begin{answer}
  设粒子的运动轨迹为一个任意的轨迹. 在轨迹上的任意一点, 粒子的速度都沿切线. 因此粒子受的 Lorentz 力沿法向. 取自然坐标系为法向和切向, 在该点上对粒子列牛二定律方程:
  \begin{align*}
    \vec{F}&=m\vec{a}\\
    \intertext{即}
    q\vec{v}\times\vec{B}&=m(\vec{a}_t+\vec{a}_n)
  \end{align*}
  Lorentz 力沿法向, 所以 $a_t=0$; 曲线运动中任意点法向加速度为 $a_t=\frac{v^2}{\rho}$\,. 得到
  \begin{equation*}
    qvB=m\frac{v^2}{\rho}\,.
  \end{equation*}
  因为 $v$ 不变, 所以得到轨迹上各点的曲率半径相同 (且非零). 这样的轨迹只有圆. (曲线上每两个无限近的点之间的轨迹都弯曲了相同的程度.)
\end{answer}
\begin{question}
  Hertz's experiment 中, 电容 $C$ 的作用是什么? 开口处不就相当于一个 $C$ 么?
\includegraphics{files/hertz-experiment}
\end{question}
\end{document}
