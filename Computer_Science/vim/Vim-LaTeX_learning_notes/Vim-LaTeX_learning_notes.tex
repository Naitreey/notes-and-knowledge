\RequirePackage[l2tabu, orthodox]{nag}
\documentclass{article}

\usepackage[a4paper]{geometry}
%文献宏包 \usepackage[style=numeric,backend=biber]{biblatex}


%中文宏包
%\usepackage[UTF8,winfonts,fancyhdr,hyperref,fntef]{ctex}
\usepackage{xeCJK}

\usepackage[pdfborder={0 0 0},colorlinks=false]{hyperref}
\usepackage{graphicx,amsmath,amssymb,bm,mathtools,subcaption,caption,cleveref}
\usepackage{examplep}


\newcommand{\command}[1]{\PVerb{#1}}
\newcommand{\syntax}[1]{\PVerb{#1}}
\newcommand{\envir}[1]{\PVerb{#1}}

%文献源 \addbibresource{•}

%设置字体
%\setmainfont[Ligatures=TeX]{Helvetica}
%\setsansfont{}
\setCJKmainfont[BoldFont={SimHei},ItalicFont={楷体}]{SimSun}%Hiragino Sans GB W3/W6
%\setCJKsansfont{}
\setCJKmonofont{幼圆}

%代码宏包
\usepackage{listings}
\lstset{columns=flexible,basicstyle=\normalfont\ttfamily,morecomment=[s]{<}{>},commentstyle=\rmfamily\itshape}

%\lstset{language=Mathematica,frame=single,texcl=true,mathescape=true}

\title{Vim--\LaTeX{} Learning Notes}
\author{Naitree Zhu}
\date{\today}

\begin{document}
\maketitle
%目录 \tableofcontents
\part{安装和基础设置}
\begin{enumerate}
\item 将所有文件解压后放入\verb`\vimfiles` 目录.
\item 对\verb`.vimrc` 和其他文件初步设置, 见: \hspace{15pt}\url{http://vim-latex.sourceforge.net/documentation/latex-suite/recommended-settings.html}.
\item install help files:
\begin{verbatim}
helptags ~/.vim/doc (for *nix)
helptags ~/vimfiles/doc (for Windows)
\end{verbatim}
\verb`~` 代表目录.
\item \verb`.vimrc` 一些设置:
\begin{enumerate}
\item set CJK encoding:
\begin{verbatim}
set encoding=utf-8
set termencoding=utf-8
set fileencodings=utf-8,ucs-bom,gb18030,gbk,gb2312,cp936
\end{verbatim}
\item set color scheme: \verb`colo darkblue` 
\item set case-insensitive search: \verb`set ignorecase`
\item set line number counting: \verb`set number`
\end{enumerate}
\end{enumerate}
以上设置的参考地址: \hspace{80pt}\url{http://vim-latex.sourceforge.net/index.php?subject=down load&title=Download}

此外, Latex--Suite reference 可以从 Vim 内查看: \verb`:help latex-suite.txt`
\part{Template}
Tex-Suite 菜单里有模板, 可以把自己的模板放到那个文件夹里.

A template file can utilize placeholders for initializing the cursor position when the template is read in and subsequent movements.
Template files can contain dynamic elements such as the time of creation of the file, by using Vim expressions.

Custom plates should be placed in \verb|$VIM/ftplugin/latex-suite/templates|.

The templates are also accessible by Ex-command: \lstinline|:TTemplate <template-name>|.
If the command is called without argument, then a list of available templates will be displayed for choice.

\part{Keyboard Shortcuts and Autocompletion}
Key mappings used by \LaTeX{}-Suite are not standard mappings in the sense that only the last character is mapped, such that you can press these combinations as slowly as you wish (and correct them).

The bracking \syntax{<++>} is used as placeholder to mark off locations where the relevant editing has to be done.
\section{\texttt{IMAP() function}}
For maps which are triggered for a given filetype, the \syntax{IMAP{}} command should be put in the filetype plugin script for that file.
For example, for \TeX{}-specific mappings, the \syntax{IMAP()} calls should go in \verb|$VIM/ftplugin/tex.vim|.
For globally visible maps, you need to use the following commands in either your \verb|~/.vimrc| or a file in your \verb|$VIM/plugin| directory. 
\begin{lstlisting}[frame=single]
augroup MyIMAPs
    au!
    au VimEnter * call IMAP('<command>','<maps>','')
augroup END
\end{lstlisting}

\section{Override \LaTeX{}-Suite macros}
To \emph{override} macros created by \LaTeX{}-Suite rather than create new macros, you should place the \syntax{IMAP()} calls in a script which gets sourced after the files in \LaTeX{}-Suite. A good method is to place a file-type plugin file in the \verb|~/.vim/after/ftplugin/| directory. (Create the \verb|after| directory and all the subfolders if necessary.) It is important to ensure the name of the file that you created conforming to the standards to get sourced on a \syntax{FileType} event.

\subsection{\texttt{IMAP()} syntax}
\begin{lstlisting}[frame=single]
call IMAP(<lhs>,<rhs>,<ft> [,<phs>,<phe>])
\end{lstlisting}

The usages of its arguments are listed as followings.
\begin{description}
  \item[\textit{lhs}] When using \syntax{IMAP} function, only the last character of the \lstinline|<lhs>| is actually mapped, although the effect is that the whole word is mapped. If two mappings end in a common \lstinline|<lhs>|, the longer mapping is used. For example, two mappings: \syntax{BarFoo} and \syntax{Foo}.
  \item[\textit{rhs}] This is the expansion obtained when typing \lstinline|<lhs>|. If \lstinline|<rhs>| contains special characters in Insert mode, the \lstinline|<rhs>| should be enclosed in a pair of doulbe quote marks \verb|" "| to become a string. Commands and special characters in this string should therefore escaped by prefixing \verb|\| character. (If \lstinline|<rhs>| is enclosed normally in single quote marks \verb|' '|, the escaping is not needed.)
  \item[\textit{ft}] The filetype for which the map is active. If empty, the map is applied for all filetypes. A filetype specific map will always take precedence.
  \item[\textit{phs,phe}] These arguments are used to specify which pair of characters should be used as placeholders in \lstinline|<rhs>|.
\end{description}

For example,
\begin{lstlisting}[breaklines=true,deletecomment={[s]{<}{>}}]
call IMAP('*EEQ',"\\begin{equation*}\<CR><++>\<CR>\\end{equation*}<++>",'tex')
\end{lstlisting}

\section{Insert Package}
输入 package name 后在 Inset mode 下按 F5 可以 autocomplete. 在 insert mode 下 Ctrl+j 光标跳至下一个 placeholder.
\section{Insert Environment} 
\subsection{Method 1: General method}
输入 environment 名字后 Insert mode 按 <F5> autocomplete, 也可以在 Normal or Insert mode 先按 <F5> 后选择或输入所需的 environment.

\subsection{Method 2: Us \texttt{IMAP} function to map letter sequences}
\label{sec:IMAPmapEnvironment}
The default letter sequences created by \texttt{IMAP} function to insert environment conform the following rules:
\begin{enumerate}
  \item All environment mappings begin with \verb|E|;
  \item If the name of environment can be split into 2 words, the initials of these words are used as the next 2 letters;
  \item If the name cannot be split into 2, the first 2 letters of environment name are used;
  \item If there are two environments whose names have idential first 2 letters, then the first and last letters of these environments are used.
\end{enumerate}

\section{Enclose text in environment}
\subsection{Method 1: \texttt{<F5>}}
Visually select a portion of text and press \syntax{<F5>}. This will prompt you with a list of environments and you can also enter other environments not listed.
\subsection{Method 2: Comma + 2 letters}
Visually select a portion of text and press a sequence of characters starting with comma (\verb|,|), following with the same combinations of letters used in \texttt{IMAP} environment mappings (\cref{sec:IMAPmapEnvironment}).
\section{Change environment}
\syntax{<S-F5>} in normal mode detects which environment the cursor is presently located in and prompts you to replace it with a new one.

\section{Insert commands}
Pressing \syntax{<F7>} when the cursor is on white-space will prompt a list of commands for your choice. You can also enter other command.
Pressing \syntax{<F7>} while the cursor is touching a word will insert a command formed from the word touching the cursor. For certain commands, \LaTeX{}-Suite will expand them to a special form.

Custom expansions of commands can also be defined.

\section{Enclose text in a command}
Visually select a portion of text and press \syntax{<F7>}.
\section{Change command}
In insert mode, place the cursor in the argument of a command and press \syntax{<S-F7>}.

\section{Greek Letter Mappings}
Lower case: \syntax{`a} to \syntax{`z} expand to \command{\alpha} through \command{\zeta}.

Upper case: \cref{tab:uppercaseMap}.
\begin{table}[htbp]
  \centering
  \begin{tabular}{ll}
    \syntax{`D}&\command{\Delta}\\
    \syntax{`F}&\command{\Phi}\\
    \syntax{`G}&\command{\Gamma}\\
    \syntax{`Q}&\command{\Theta}\\
    \syntax{`L}&\command{\Lambda}\\
    \syntax{`X}&\command{\Xi}\\
    \syntax{`Y}&\command{\Psi}\\
    \syntax{`S}&\command{\Sigma}\\
    \syntax{`U}&\command{\Upsilon}\\
    \syntax{`W}&\command{\Omega}
  \end{tabular}
  \caption{Uppercase letter mappings.}
  \label{tab:uppercaseMap}
\end{table}

\section{Miscellaneous Insert mode key mappings}
Insert mode miscellaneous key mappings are listed in \cref{tab:miscellaneous}.
\begin{table}
  \centering
  \begin{tabular}{ll}
    \syntax{`^}&\command{\hat{<++>}<++>}\\
    \syntax{`_}&\command{\bar{<++>}<++>}\\
    \syntax{`6}&\command{\partial}\\
    \syntax{`8}&\command{\infty}\\
    \syntax{`/}&\command{\frac{<++>}{<++>}<++>}\\
    \syntax{`=}&\command{\equiv}\\
    \syntax{`\ }&\command{\setminus}\\
    \syntax{`&}&\command{\wedge}\\
    \syntax{`<}&\command{\le}\\
    \syntax{`>}&\command{\ge}\\
    \syntax{`~}&\command{\tilde{<++>}<++>}\\
    \syntax{`2}&\command{\sqrt{<++>}<++>}\\
    \syntax{`I}&\command{\int_{<++>}^{<++>}<++>}
  \end{tabular}
  \caption{Miscellaneous Insert mode key mappings.}
  \label{tab:miscellaneous}
\end{table}

\section{Visual mode brackets mappings}
Visual mode brackets mappings are listed in \cref{tab:visualBrackets}.
\begin{table}[htbp]
  \centering
  \begin{tabular}{ll}
      \syntax{`(}&enclose visual selection in \command{\left(  \right)}\\
      \syntax{`[}&enclose visual selection in \command{\left[  \right]}\\
      \verb|`{|&enclose visual selection in \command{\left\{  \right\}}\\
      \verb|`$|&enclose visual selection in \command{$ $}
  \end{tabular}
  \label{tab:visualBrackets}
  \caption{Visual mode brackets mappings.}
\end{table}

\section{Smart key mappings}
\begin{description}
  \item[Smart quotes.] Pressing \verb|"| will insert \verb|``| or \verb|''| depending on making an guess about whether an open or closing quote mark is intended.
  \item[Smart Dots.] Pressing \verb|...| results in \command{\ldots} outside math mode and \command{\cdots} in math mode.
\end{description}

\section{Miscellaneous \texttt{Alt} key mappings}
To use \texttt{Alt} key to construct key mappings,
\begin{lstlisting}
set winaltkeys=no
\end{lstlisting}
must be set in your \verb|$VIM/ftplugin/tex.vim| file.

\subsection{\texttt{<Alt-i>}}
\syntax{<Alt-i>} maps to an \command{\item} command at the cursor location.
The style of the \command{\item} command is dependent on the enclosing environment.

\begin{center}
  \begin{tabular}[c]{ll}
  \envir{itemize}&\command{\item}\\
  \envir{enumerate}&\command{\item}\\
  \envir{description}&\command{\item[<+label+>] <++>}\\
  \envir{theindex}&\command{\item}
  \end{tabular}
\end{center}

The \syntax{<Alt-i>} will insert nothing if used outside proper environment.

\section{Autocompletion}
在 \verb|.vimrc| 中添加一些定义:
\begin{verbatim}
let g:Tex_Com_verb = "\\verb|<++>|<++>"
let g:Tex_Com_frac = "\\frac{<++>}{<++>}<++>" 
let g:Tex_Com_newcommand = "\\newcommand{<++>}[<++>]{<++>}<++>"
let g:Tex_Com_renewcommand = "\\renewcommand{<++>}{<++>}<++>"
let g:Tex_Com_sin = "\text{sin}\,<++>"
let g:Tex_Com_cos = "\text{cos}\,<++>"
let g:Tex_Com_ln = "\text{ln}\,<++>"
let g:Tex_Com_d = "\mathrm{d}^[<++>]<++>"
\end{verbatim}

Alt Key Macros:

Firstly, disable \verb|<Alt>| to focus to the menu in Vim: Add \verb|set winaltkeys=no| in the file \verb|$VIM/ftplugin/tex.vim|. 

Then use \verb|Alt-l| in Insert mode to expands to one of the following depending on the character just before the cursor location.
\begin{tabular}[h]{lr}
  \hline
  \verb|(| & \verb|\left( <++> \right)| \\
  \hline
  \verb|[| & 同上 \\
  \hline
  \verb`|` & 同上 \\
  \hline 
  \verb|{| & 同上 \\
  \hline
  \verb|<| & \verb|\langle <++> \rangle| \\
  \hline
  \verb|q| & \verb|\lefteqn{<++>}<++>| \\
  \hline
\end{tabular}
\section{Folding}
在 normal mode 下 \verb|\rf| 可以折叠全文除了光标处以外的所有结构. 想要展开某个结构, 光标放在该行按 za. 在一个展开的部分按 \verb|za| 可以折叠该结构.
\subsection*{Folding Settings}
在 \verb|.vimrc| 中添加以下赋值语句让 vimlatex 不折叠 \verb|paragraph| 和 \verb|item| (默认时有这两项):
\begin{verbatim}
let Tex_FoldedSections = 'part,chapter,section,%%fakesection,
subsection,subsubsection'
let Tex_FoldedMisc = 'preamble,<<<'
\end{verbatim}
添加以下赋值语句防止折叠不该折叠的环境:
\begin{verbatim}
let Tex_FoldedEnvironments = 'comment,equation,align,gather,figure,
table,thebibliography,keywords,abstract,titlepage'
\end{verbatim}
\section{Insert Reference}
光标在 \verb`\ref{}` 中间时, 在 insert mode 下按 F9, 出现所有 Label 列表窗口. 将光标置于所需 label 行头然后回车, 就输入了 label.

为了使这个正常工作, 需要设定:
\begin{enumerate}
\item 安装 \verb`grep` 程序.
\item 将 grep 添加为系统的 environment variables 列表中系统变量 PATH 变量值之一. PATH 的值包含所有需要 PATH 声明的程序目录地址, 两程序之间要用 ``;'' 隔开. 环境变量需要重启生效.
\item 保证 \verb`.vimrc` 中有这个设定: \verb`set grepprg=grep\ -nH\ $*`
\item 安装 python 2.7版本, 不要安装比较高的那个3.x版本. 注意 Vim 与 python 需要相同的 32/64-bit 版本. 
\end{enumerate}
\part{Compile}
\verb`\ll` 将编译, 不过编译前可能需要先保存. 默认的是 LaTeX, 生成 dvi.
\subsection*{compile xelatex}
\begin{enumerate}
\item 更改默认输出格式. \verb`.vimrc` 中添加: \verb`let g:Tex_DefaultTargetFormat = "pdf"`. 这样输入\verb`\ll` 和 \verb`\lv` 将输出 pdf 和打开 pdf. 注意在 mac 中默认就是 pdf, 不需要这个设置.
\item 设置 pdf 编译器:\hfill\\
  \verb`let g:Tex_CompileRule_pdf 'xelatex -interaction=nonstopmode $*'`
\end{enumerate}
\subsection*{设置xelatex 与 pdflatex 转换}
在 \verb`.vimrc` 中设置:
\begin{verbatim}
"switch set pdflatex
function SetpdfLaTeX()
	let g:Tex_CompileRule_pdf = 'pdflatex -interaction=nonstopmode -file
-line-error-style $*'
endfunction
map <Leader>lp :<C-U>call SetpdfLaTeX()<CR>
"switch set xelatex
function SetXeLaTeX()
	let g:Tex_CompileRule_pdf = 'xelatex -interaction=nonstopmode -file
-line-error-style $*'
endfunction
map <Leader>lx :<C-U>call SetXeLaTeX()<CR>
\end{verbatim}
然后在默认 xelatex 下, \verb`\lp` 切换至 pdflatex, \verb`\lx` 切换回 xelatex, 再用\verb`\ll` 来编译.
\subsection*{IgnoredWarnings}
禁止忽略警告: \verb`let g:Tex_IgnoreLevel 0`
\part{Debug}
出现错误后, 跳出两个窗口, 一个是 Error list, 另一个是 log. 光标在 Error list 中的某个 error 上回车, 就跳到了 error 所在处. (这个回车反跳的功能有问题.)
\part{View PDF}
按照\verb`.vimrc` 中默认输出格式设定, normal mode 下 \verb`\lv` 打开 dvi/pdf viewer.
\subsection*{view pdf}
\begin{enumerate}
\item 下载 SumatraPDF. 在系统环境变量 PATH 下添加程序目录.
\item 在 \verb`.vimrc` 中设置 pdf viewer:\verb`let g:Tex_ViewRule_pdf = "SumatraPDF"`
\end{enumerate}
\subsection*{forward search \& inverse search with SumatraPDF}
在 Vim 中 normal mode 下 \verb`\ls` 可以 forward search. forward \& inverse search 需要 pdf viewer 的支持.
Windows 下利用 SumatraPDF 可以进行 forward search \& inverse search. 
inverse search 比较容易实现, 只需要在 \verb`.vimrc` 中对各 engine 和 pdf viewer 进行设置即可.
pdf viewer 设置:
\begin{verbatim}
let g:Tex_ViewRule_pdf = 'SumatraPDF -reuse-instance -inverse-search "g
vim -c \":RemoteOpen +\%l \%f\""'
\end{verbatim}

在所有 engine 的设置中添加两个选项: \verb`-synctex=1 -src-specials`.

forward search 在 Windows 下如果不懂得原理的话, 难以实现. 因为网上现成的方案已经十分过时. 在新版本下已经行不通. 根据我的搜索结果, 我目前理解的原理框架是这样的:\hfill\\
由于vim-latex 下的 compiler.vim 默认只提供了实现 dvi 格式的 forward searching function, 需要对这部分改写, 以满足与相关程序连接的功能. 流程我估计是, 在 Vim 下输入 \verb`\ls` 后编译器会读入光标的位置等相关信息, 
在forward searching function 中运算后再输入至一个外部程序, 该程序可以设法调用 \verb`dde.exe` 来输出一个同步信息给 pdf viewer, 从而把 Vim 中的光标值转化为 pdf 中的位置.

下面就是那个已经过时的方法. 这也侧面地再次印证了一件事情: 远离 Windows! 在 MacOS X 下这一切设置都没有这么地曲折.
\begin{enumerate}
  \item 下载 ``Python for Windows'' extension: \hspace{15pt}\url{http://sourceforge.net/projects/pywin32/files/pywin32/}, 注意匹配 python 版本.
  \item 下载配置方法和文件. 见\hspace{15pt}\url{http://william.famille-blum.org/blog/static.php?page=static081010-000413} 最下部 Vim 部分.
\item 复制 fwdsumatra.py 和 fwdsumatra.diff 至 ftplugin/latex-suite folder.
\item 下载 Cygwin installer.  打开后搜索 pack, 安装相关的所有 package. 打开 terminal, cd + ~Vim/vimfiles/ftplugin/latex-suite, 将操作目录移动到该文件目录. 在 terminal 中写:\verb`patch -p0 -c -b compiler.vim fwdsumatra.diff`.
\item 在 \verb`.vimrc` 中修改 xelatex 和 pdflatex 的值为:
  \begin{verbatim}
  let g:Tex_CompileRule_pdf = 'pdflatex --synctex=-1 -src-specials -
  interaction=nonstopmode $*' 
  \end{verbatim}
  修改 pdf viewer 的值为:
  \begin{verbatim}
  let g:Tex_ViewRule_pdf = 'SumatraPDF -inverse-search "gvim -c \":R
  emoteOpen +\%l \%f\""'
  \end{verbatim}
\item 下载 dde.exe 置于 ftplugin/latex-suite folder.
\end{enumerate}
在 CTeX 论坛上找到了一个运用系统批处理文件绕过 python 来实现正向搜索的帖子, 是在上面这个过时的方法基础上作出一些改动, 将 forward search function 的输入转向批处理. 我不懂原理, 不做分析. 
值得注意的是, 这个方法在 patch 时是借用已过时的函数来做的, 因此 compiler.vim 不能使用 2013年版本, 需使用稍老的 2010 年版本, 方可 patch 成功. 该笔记所在目录有这个压缩包. 但为了长远考虑, 具体实现方法为:

写 \verb|sumatrapdfview.diff| 文件:
{\tiny
\begin{verbatim}
--- compiler.vim	2009-03-12 00:11:26 +0800
+++ compiler-liangzi.vim	2009-04-21 01:17:03 +0800
@@ -322,6 +322,11 @@
 "           gvim --servername LATEX --remote-silent +%l "%f"
 "      For inverse search, if you are reading this, then just pressing \ls
 "      will work.
+"
+" Use DDE to communicate with SumatraPDF, hence calling a Python
+" file with the DDE gestion. Use same s:path as used for
+" calling outline.py in texviewer.vim
+let s:path = expand('<sfile>:p:h')
 function! Tex_ForwardSearchLaTeX()
 	if &ft != 'tex'
 		echo "calling Tex_ForwardSeachLaTeX from a non-tex file"
@@ -344,10 +349,23 @@
 	
 	" inverse search tips taken from Dimitri Antoniou's tip and Benji Fisher's
 	" tips on vim.sf.net (vim.sf.net tip #225)
-	if (has('win32') && (viewer == "yap" || viewer == "YAP" || viewer == "Yap"))
+	if (has('win32') && (viewer == "yap" || viewer == "YAP" || viewer == "Yap" || viewer =~? "^sumatrapdf"))
+
+		if (viewer == "yap" || viewer == "YAP" || viewer == "Yap")
 
-		let execString = 'silent! !start '. viewer.' -s '.line('.').expand('%').' '.mainfnameRoot
+			let execString = 'silent! !start '. viewer.' -s '.line('.').expand('%').' '.mainfnameRoot
 
+		elseif viewer =~? "^sumatrapdf"
+			
+			" SumatraPDF requires the .tex file path to be given relatively to
+			" the main file path, hence the substitute
+			"let relativeFile=substitute(expand("%:p"), Tex_GetMainFileName(':p:h').'/', '','')
+			"let execString = 'silent! !'.s:path.'/fwdsumatra.py "'.mainfnameFull.'.'.s:target.'" "'.relativeFile.'" '. line('.')
+                         let relativeFile=substitute(expand("%:p"), "\\/", '\',"g")
+                         let pdfname=substitute(mainfnameFull.'.'.s:target, "\\/", '\',"g")
+			let execString = 'silent! !start "'.s:path.'/sumatrapdfview.bat" "'.pdfname.'" "'.relativeFile.'" '. line('.')
+
+		endif
 
 	elseif (has('macunix') && (viewer == "Skim" || viewer == "PDFView" || viewer == "TeXniscope"))
 		" We're on a Mac using a traditional Mac viewer

\end{verbatim}
}
写 \verb|sumatrapdfview.bat| 文件:
{\tiny
\begin{verbatim}
@echo off
set EDTROOT=%~dp0
set file=%~1
set line=%~3
set src=%~2
set _d=%TEXROOT%\tools\sumatrapdf
set _r=%TEXROOT%
	IF   [%1]==[]   GOTO   usage
	IF   [%2]==[]   GOTO   usage
	IF   [%3]==[]   GOTO   usage
start /max "" SumatraPDF -reuse-instance -inverse-search "gvim.bat -c \":RemoteOpen +%%l %%f\""
ping 127.0.0.1 -n 1 1>nul
"%EDTROOT%dde.exe" SUMATRA SUMATRA control "[ForwardSearch("%file%","%src%",%line%,0,0,1)]"
exit /B 0

:usage
echo Usage: %0 pdffile line_no srcfile ==) Forward search pdf file with Sumatrapdf
echo Note: Only Sumatrapdf supports pdf forward search!
exit /B 0
\end{verbatim}
}
将以上两文件连同 \verb|dde.exe| 三文件一起放入 /ftplugin/latex-suite/ 目录.
在 Cygwin 中用 \verb|patch -p0 compiler.vim sumatrapdfview.diff| 命令将 diff 文件 patch 给 compiler.vim , 至此设置完毕. 
这样就可以在 SumatraPDF 中双击 inverse search 回 tex 源文件, 在 tex 源文件中 \verb|\ls| forward search 至 pdf 中对应位置. 还需要注意一点, 为了两文件能找到对方, 它们所在的目录不能含有空格, 更不能含有 CJK 字符.
\part{biber}
为了使用 biber, 需要 arara.
下载 arara 程序, 在 \url{https://github.com/cereda/arara} 页面上把整个包下载下来. 安装 installer, 让程序自动把目录加入 PATH. 

arara 作为 IDE 与 TeX Distribution 中各种程序的中介, 通过 Vim 将指令发送给 arara, arara 去指挥操作流程并从 TeX 程序反馈运行结果信息和完整报告.
Vim 只需和 arara 沟通, 故在 \verb|.vimrc| 中设置 arara 作为指令接收对象, 在 \verb|.tex| 源文件中添加对 arara 的操作指令.具体如下:
\begin{enumerate}
  \item 在 \verb|.vimrc| 中添加一条切换至 arara 控制的函数, 设置使用 \verb|\la| 来执行这个函数. 
    \begin{verbatim}
"switch to arara control
function SetAraraControl()
	let g:Tex_CompileRule_pdf = 'arara -l -v $*'
endfunction
map <Leader>la :<C-U>call SetAraraControl()<CR>
    \end{verbatim}
    其中, \verb|-l| 让 arara 生成所有命令的运行结果报告, \verb|-v| 使 arara 不使用 silent mode, 从而可以返回后续程序生成的信息(和要求)给 terminal(这里则是IDE).
  \item 在 \verb|.tex| 源文件中添加 arara directive.
    \begin{verbatim}
% arara: xelatex: {action: nonstopmode,synctex: yes}
% arara: biber
% arara: xelatex: {action: nonstopmode,synctex: yes}
% arara: xelatex: {action: nonstopmode,synctex: yes}
    \end{verbatim}
    根据 TeX 生成 pdf 的流程, 需要按照这样的顺序运行 4 遍: 第一遍生成 pdf, 第二遍生成文献相关部分, 第三遍结合第二遍结果重新生成 pdf 从而加入交叉引用和文献相关内容, 第四遍重新生成 pdf 检查一切.
    再结合 arara 的语法要求, 得到这 4 条 directive. 其中, \verb|action| 值为 \verb|nonstopmode|, 使 xelatex 连续编译遇到错误尽量忽略, 最后显示所有的错误和警告; \verb|synctex| 值为真(\verb|yes|), 使得 xelatex 生成可以进行 forward \& inverse search 的必要文件(注意这时不再需要 Vim-LaTeX 要求的 \verb|-src-specials| 选项). 
\end{enumerate}
由于现在运行了 4 遍, 带有 \verb|--verbose| flag 的 arara 将返回 4 遍下来所有的警告、错误等信息给 terminal, 对于 Vim 而言则显示在 quickfix list 窗口. 只要将各遍产生的信息分清楚别弄混就好, 这样所有信息都看得到还便于查错. 
TeX engine 会生成一个最终运行结果的 \verb|filename.log| (之前的都被覆盖了), 同时 arara 会生成一个 \verb|arara.log|, 包含运行过程中的从 engine 中反馈出来的所有信息.
\part{Tricks}
\paragraph{double quotes}
vim-latex 默认将双引号 \verb|"| 映射为 \verb|``''|, 要输入双引号的话可以用 Vim 的自带纯字符输入命令 <Ctrl-v> 再输入 \verb|"|. (注意在 Windows 下, 要用 <Ctrl-q> 代替 <Ctrl-v>.)

\part{\texttt{vimrc} 文件}
这里列出 \verb|.vimrc| 里的所有设置:
{\tiny
\begin{verbatim}
set nocompatible
source $VIMRUNTIME/vimrc_example.vim
source $VIMRUNTIME/mswin.vim
behave mswin

set diffexpr=MyDiff()
function MyDiff()
  let opt = '-a --binary '
  if &diffopt =~ 'icase' | let opt = opt . '-i ' | endif
  if &diffopt =~ 'iwhite' | let opt = opt . '-b ' | endif
  let arg1 = v:fname_in
  if arg1 =~ ' ' | let arg1 = '"' . arg1 . '"' | endif
  let arg2 = v:fname_new
  if arg2 =~ ' ' | let arg2 = '"' . arg2 . '"' | endif
  let arg3 = v:fname_out
  if arg3 =~ ' ' | let arg3 = '"' . arg3 . '"' | endif
  let eq = ''
  if $VIMRUNTIME =~ ' '
    if &sh =~ '\<cmd'
      let cmd = '""' . $VIMRUNTIME . '\diff"'
      let eq = '"'
    else
      let cmd = substitute($VIMRUNTIME, ' ', '" ', '') . '\diff"'
    endif
  else
    let cmd = $VIMRUNTIME . '\diff'
  endif
  silent execute '!' . cmd . ' ' . opt . arg1 . ' ' . arg2 . ' > ' . arg3 . eq
endfunction

"Vim-LaTeX settings
filetype plugin on
set shellslash
set grepprg=grep\ -nH\ $*
filetype indent on
let g:tex_flavor='latex'

"set compile engine and parameters delivered to engine
let g:Tex_CompileRule_dvi = 'latex --interaction=nonstopmode $*'
let g:Tex_CompileRule_pdf = 'xelatex --interaction=nonstopmode -synctex=1 -src-specials $*'

"switch to pdflatex
function SetpdfLaTeX()
	let g:Tex_CompileRule_pdf = 'pdflatex --interaction=nonstopmode -synctex=1 -src-specials $*'
endfunction
noremap <Leader>lp :<C-U>call SetpdfLaTeX()<CR>

"switch to xelatex
function SetXeLaTeX()
	let g:Tex_CompileRule_pdf = 'xelatex --interaction=nonstopmode -synctex=1 -src-specials $*'
endfunction
noremap <Leader>lx :<C-U>call SetXeLaTeX()<CR>

"switch to arara control
function SetAraraControl()
	let g:Tex_CompileRule_pdf = 'arara -l -v $*'
endfunction
noremap <Leader>la :<C-U>call SetAraraControl()<CR>

"set default output format as PDF
let g:Tex_DefaultTargetFormat = 'pdf'

"set PDF viewer
let g:Tex_ViewRule_pdf = 'SumatraPDF -reuse-instance -inverse-search "gvim -c \":RemoteOpen +\%l \%f\""'

"(re)new vim-latex command shortcuts
let g:Tex_Com_verb = "\\verb|<++>|<++>"
let g:Tex_Com_frac = "\\frac{<++>}{<++>}<++>"
let g:Tex_Com_tfrac = "\\tfrac{<++>}{<++>}<++>"
let g:Tex_Com_dfrac = "\\dfrac{<++>}{<++>}<++>"
let g:Tex_Com_newcommand = "\\newcommand{<++>}[<++>]{<++>}<++>"
let g:Tex_Com_renewcommand = "\\renewcommand{<++>}[<++>]{<++>}<++>"
let g:Tex_Com_tbf = "\\textbf{<++>}<++>"
let g:Tex_Com_ttt = "\\texttt{<++>}<++>"
let g:Tex_Com_tsf = "\\textsf{<++>}<++>"
let g:Tex_Com_tit = "\\textit{<++>}<++>"
let g:Tex_Com_tex = "\\TeX{}<++>"
let g:Tex_Com_pdftex = "pdf\\TeX{}<++>"
let g:Tex_Com_latex = "\\LaTeX{}<++>"
let g:Tex_Com_latexe = "\\LaTeXe{}<++>"
let g:Tex_Com_pdflatex = "pdf\\LaTeX{}<++>"
let g:Tex_Com_xetex = "\\XeTeX{}<++>"
let g:Tex_Com_xelatex = "\\XeLaTeX{}<++>"
let g:Tex_Com_luatex = "Lua\TeX{}<++>"
let g:Tex_Com_amslatex = "\\AmS-\\LaTeX{}<++>"
let g:Tex_Com_metapost = "\\METAPOST{}<++>"
let g:Tex_Com_metafont = "\\METAFONT{}<++>"
let g:Tex_Com_mbb = "\\mathbb{<++>}<++>"
let g:Tex_Com_binom = "\\binom{<++>}{<++>}<++>"
let g:Tex_Com_stx = "\\syntax{<++>}<++>"

"show all warnings
let g:Tex_IgnoreLevel = 0

"set folded sections
let Tex_FoldedSections = 'part,chapter,section,%%fakesection,subsection,subsubsection'

"set folded misc
let Tex_FoldedMisc = 'preamble,<<<'

"set foled environments
let Tex_FoldedEnvironments = 'comment,equation,align,gather,figure,table,thebibliography,keywords,abstract,titlepage'

"set CJK character encoding
set encoding=utf-8
set termencoding=utf-8
set fileencodings=utf-8,ucs-bom,gb18030,gbk,gb2312,cp936

"pathogen plugin
execute pathogen#infect()

"solarized colorscheme
syntax enable
set background=dark
colorscheme solarized


"set case-insensitive search
set ignorecase

"set line number count
set number

"set GUI font
if has('gui_running')
  set guifont=Meslo_LG_L:h11:b:cANSI
endif

"hide gui menu bar "set guioptions=egmLt
set guioptions=egt

"set wildmenu to show matches, so that <C-d> is not generally needed.
set wildmenu

"remap <Y> to the more logical <y$>
map Y y$

"mathematica syntax hightlighting in solarized colorscheme
let g:mma_highlight_option = "solarized"
"Smart Conceal for Mathematica
let g:mma_candy = 1

"stop Voom remapping <Ctrl-i>
let g:voom_tab_key = ""

"set initial window size
if has("gui_running")
	set lines=30 columns=120
endif

"keep 200 lines of command-line history
set history=200

"remap <C-p> & <C-n> as <Up> and <Down>. To autocomplete command-line, use
"<Tab> and <S-Tab>
cnoremap <C-p> <Up>
cnoremap <C-n> <Down>
\end{verbatim}
}
\end{document}
