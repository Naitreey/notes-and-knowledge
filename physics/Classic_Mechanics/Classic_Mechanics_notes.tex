\documentclass{article}
\begin{document}
\part{Q \& A}
\begin{question}
  力的合成与矢量的加法运算为什么服从平行四边形定则?
\end{question}
\begin{answer}
  \begin{enumerate}
    \item 力的合成服从平行四边形定则的原因是, 这是实验发现的结果: 即, 从实验上发现, 有这样的一个力, 它与那些不同的力一起作用于物体上时对物体运动状态的影响效果相同.
    \item 从实验上得到的力的这种等效性质, 可以看作是数学上按照平行四边形定则定义矢量加法的动机. 并且在数学上如此定义后, 便可以把力的运算统一在矢量运算的范畴之下.
    \item 至于从根本上, 为什么力恰恰具有这样的满足平行四边形定则的合成性质呢? 我认为这从现阶段只能看作是宇宙的性质, 或者说是空间的性质. 注意到平行四边形定则的基础是勾股定理. 这似乎只能看作是这个空间的性质所导致的.
  \end{enumerate}
\end{answer}

\begin{question}
  手提着重物时, 拉力不做功, 为什么会感到累?
\end{question}
\begin{answer}
  这需要从生物物理的角度分析. 手臂拉着重物, 拉力是由于无数的肌肉纤维紧张而产生的.
  
  从理想状态来讲, 初始的拉伸过程中各肌肉纤维的拉力有位移.  如果手臂上这些肌肉在紧张后一直维持这个状态而保持不动, 则没有其他位移在后续过程中产生, 这样手臂就不再做功, 不会持续不断地消耗能量.

  但实际上, 手臂上的一个肌肉纤维不会一直维持着紧绷状态, 在一段时间后必然要放松. 为了保持拉住物体, 其他的肌肉纤维就要替代它来提供一份拉力. 后者在从松驰至紧张状态中拉力要有位移要做功. 因此从宏观角度看, 每时每刻都有新的肌肉纤维代替原来的纤维来拉住物体, 而每个肌肉纤维在初始的紧张过程中都要做一次功. 因此总是有能量消耗. (这种化学能转化为功和热, 因此手臂还会慢慢变热.)
\end{answer}

\begin{question}
  人是如何向前走的? 这个过程中谁做功? 
\end{question}
\begin{answer}
  后脚向后蹬地, 脚给地的力向后, 脚与地之间的静摩擦力向前. 这两个力平衡, 防止向后打滑, 脚保持静止. 腿上的肌肉向前发力, 给身体向前的加速度. 人就运动了起来. 这个过程中身上的各个肌肉用力并做功, 消耗身体化学能. 地面的静摩擦力没有位移不做功.
  那么, 把人看作质点系, 合外力改变改变质点系的运动状态是怎么回事?
\end{answer}

\begin{question}
  一个力的功为什么定义为 $\vec{F}\cdot\dd\vec{r}$? 为什么物体动能定义为 $\frac{1}{2}mv^2\,$? 为什么合外力做了多少功物体动能就增加了多少?
\end{question}
\begin{answer}
  \begin{enumerate}
    \item 功定义为力与位移之积, 是人们在多年实践中总结出来的定义. 人们发现这样定义的功, 能够体现能量的转移, 且在数学上恰好与其他概念都吻合. (Ask Stackexchange!)
    \item 功的定义中的 $\cos\theta$, 源于实验事实: 当物体的轨迹确定以后, 作用在该物体上的某一个力对物体运动状态改变的贡献, 与一个轨迹方向上、大小等于 $F\cos\theta$ 的力相同. 因此力 $\vec{F}$ 对物体的作用有效的那部分就是 $F\cos\theta$ 而已. 所以定义力对物体的功为 $\vec{F}\cdot\dd\vec{r}$.
    \item 根据另一个实验事实, 牛二定律, 一个物体所受合外力决定它的运动状态如何变化. 将两个事实结合可以得到一个新的量:
      \begin{equation*}
	\vec{F}\cdot\dd\vec{r}=m\D{\vec{v}}{t}\cdot\dd\vec{r}=m\vec{v}\cdot\dd\vec{v}=\dd\left( \frac{1}{2}mv^2 \right)\,,
      \end{equation*}
      这说明合外力对物体的功, 即物体运动状态的改变量可以由 $\frac{1}{2}mv^2$ 的前后差值体现. 功是转移给物体的能量, 所以 $\frac{1}{2}mv^2$ 表示由于物体处于某个运动状态时所具有的能量. 所以这就是动能.
    \item 可见动能的定义和动能与功的关系是结合两个实验事实后一起得到的.
  \end{enumerate}
\end{answer}

\begin{question}
  经典力学中, 不同参考系 (包含惯性和非惯性系) 中测到的同一个力为什么相同? 这是日常经验的结果还是推出来的?
\end{question}
\begin{answer}
  基础假设:
  \begin{enumerate}
    \item 不同参考系中 (惯性或非惯性) 时间的流逝同步. 由此可以推出不同参考系之间的位置、速度、加速度的伽利略变换.\\
      解释: 这个假设来源于直接的日常经验. 很难想象一个运动物体上的时间会与不运动时不同.
    \item 伽利略的相对性原理: 力学规律在任何惯性系中都是相同的.\\
      解释: 不同惯性系中的物体运动遵守的力学规律相同, 这是十分自然、十分 intuitive 的假设. 这是因为, 可以想象, 在一个惯性系中所做的任何力学实验都无法判断这个惯性系的运动状态.
    \item 不同参考系中 (惯性或非惯性) 物体的质量不变.\\
      解释: 这个假设来源于直接的日常经验. 很难想象一个物体运动起来了以后它的质量 (即保持自身运动状态的能力) 就会不同.
  \end{enumerate}
  分析:
  \begin{enumerate}
    \item 解释不同惯性系中测到的同一个力是相同的. 力由动量的变化率来定义和测量, 将两个惯性系中的力写出:
      \begin{align*}
	\F&=m\D{\v}{t}\\
	\F'&=m\D{\v'}{t}=m\D{\v-\v_0}{t}=m\D{\v}{t}
      \end{align*}
      因此两个惯性系中测出的同一个力是相同的. 以上推导中用到了时间相同假设 (及伽利略速度变换)、力学定律 (牛二定律) 相同假设、质量不变假设.
    \item 解释非惯性系中测到的力也与任何惯性系中测到的同一个力相同. 还无法解释.
  \end{enumerate}
\end{answer}
\end{document}

